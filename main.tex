\documentclass[a4paper, 12pt, openany, uplatex]{jsbook}
% uplatex オプションを指定し、ユニコード対応に。ただし、uplatex&pbibtexでコンパイルすること。
% onesideはすべてのページが同じ体裁で出力される
% twosideは(両面印刷を前提として)、偶数ページと奇数ページが異なる体裁で出力される

% 修論本体と表紙で共通で必要となる設定
\input{setting/preamble}

% すべてのTS1コマンドを有効
\usepackage[full]{textcomp}
% 画像の取り扱いに必要
\usepackage[dvipdfmx]{graphicx,color}
% 表でセルを複数列で結合する
\usepackage{multicol}
% 数式の機能を拡張
\usepackage{amsmath,amssymb}
\usepackage[amsmath,thmmarks]{ntheorem}
% Time系フォント
\usepackage{newtxtext}
% 英語のフォント指定 Helvetica
\usepackage{helvet}
% 太字のベクトル
\usepackage{bm}
% 複数の図を表みたいにまとめる
\usepackage{caption}
\usepackage{subcaption}
\captionsetup{compatibility=false}
%%%%%%%%%%%%%%%%%%%%%%%%%%%%%%%%%%%%%%%%%%%%%%%%%%%%%%%%%%%%%%%%%%%%%%%%
% 行番号を表示する。添削時のみに使い、事務提出版ではコメントアウトする %
\usepackage{lineno}
\linenumbers
%%%%%%%%%%%%%%%%%%%%%%%%%%%%%%%%%%%%%%%%%%%%%%%%%%%%%%%%%%%%%%%%%%%%%%%%
% PDF内で外部リンクや文書内リンクを生成したい場合に使う
\usepackage[dvipdfmx]{hyperref}
% PDF内のしおりの文字化けを防ぐ
\usepackage{pxjahyper}
% フォントサイズの警告を消す
\usepackage{lmodern}
% フォントのエラーを消す
\DeclareFontShape{JT2}{hgt}{m}{it}{<->ssub*hgt/m/n}{}
\DeclareFontShape{JY2}{hgt}{m}{it}{<->ssub*hgt/m/n}{}
% 細かいエラーや古いパッケージを教えてくれる
\RequirePackage[l2tabu, orthodox]{nag}
% mythesis.styを使用する
\usepackage{mythesis}
% 複数行をコメント化
\usepackage{comment}

% newcommand を使うことで、繰り返し使う長ったらしい入力を簡単にすることができる
\makeatletter
\def\mojiparline#1{
	\newcounter{mpl}
	\setcounter{mpl}{#1}
	\@tempdima=\linewidth
	\advance\@tempdima by-\value{mpl}zw
	\addtocounter{mpl}{-1}
	\divide\@tempdima by \value{mpl}
	\advance\kanjiskip by\@tempdima
	\advance\parindent by\@tempdima
}
\makeatother
\def\linesparpage#1{
	\baselineskip=\textheight
	\divide\baselineskip by #1
}


%%------------------------------------------------------------
%% pdfに記載される内容のメイン
%%------------------------------------------------------------

% 氏名などの情報が入っているファイル。各自で編集。
% 論文題目
\reiwa{2} % 令和何年
\title{例題 \\ だよ} % タイトル
\entitle{EXAMPLE} % 英語タイトル
\StudentIdNumber{1896} % 学籍番号
\author{Your ~~ Name} % 氏名


\begin{document}

% 一行あたり文字数の指定
\mojiparline{40}
% 1ページあたり行数の指定
\linesparpage{30}


%%%%%%%%%%%%%%%%%%%%%%%%%%%
%%%       表紙,目次     %%
%%%%%%%%%%%%%%%%%%%%%%%%%%%
\frontmatter

\maketitle

\thispagestyle{plain}
\addtocounter{page}{-1}
\newpage

\mbox{}
\thispagestyle{plain}
\addtocounter{page}{-1}

\newpage

\chapter*{序論}
Abstract

% 目次
\tableofcontents

% 図目次
\listoffigures

% 表目次
\listoftables

% 紙に印刷して校正するために行間を開ける
% \setlength{\baselineskip}{1cm}

%%%%%%%%%%%%%%%%%%%%%%%%%%%
%%%       本文           %%
%%%%%%%%%%%%%%%%%%%%%%%%%%%
\mainmatter
\input{parts/1-Introduction}
\chapter{関連研究}
Related Work

\chapter{提案手法}
Method~\cite{exmaple}
\input{parts/4-Experiment}
\input{parts/5-Discussion}
\input{parts/6-Conclusion}
\input{parts/7-Appendix}
\input{parts/8-Acknowledgments}

%%%%%%%%%%%%%%%%%%%%%%%%%%%%%%%%%%%%
%%%   謝辞,参考文献,研究業績    %%
%%%%%%%%%%%%%%%%%%%%%%%%%%%%%%%%%%%%
\backmatter
%\chapter*{参考文献}
%\addcontentsline{toc}{chapter}{参考文献}

%\bibliographystyle{jecon}
\bibliographystyle{junsrt}
\bibliography{thesis}
\chapter*{研究業績一覧}
\addcontentsline{toc}{chapter}{研究業績一覧}

\section*{ポスター発表}

\begin{enumerate}
	\renewcommand{\labelenumi}{(\arabic{enumi})}
	\item ここに書く
\end{enumerate}

\section*{査読付論文}

\begin{enumerate}
	\renewcommand{\labelenumi}{(\arabic{enumi})}
	\item ここに書く
\end{enumerate}

\end{document}
